\documentclass[a4paper, 11pt]{article}
\usepackage[utf8]{inputenc}
\usepackage[italian]{babel}

\title{La criminalità organizzata}
\author{Jacopo Nicora}
\date{\today}

\begin{document}

\maketitle

\newpage

\tableofcontents

\newpage

\input{Introduzione_conclusione_bibliografia/Introduzione.tex}

\newpage

\input{Capitoli/Capitolo2.tex}
Nel periodo della globalizzazione, ovvero nel periodo dove, grazie a tutte le nuove tecnologie e innovazioni (specialmente nel campo della telematica), avviene l’unificazione dei mercati a livello globale, la criminalità organizzata svolge sicuramente il ruolo di antagonista sociale più temuto e più pericoloso. Per comprendere la sua importanza e la sua insidiosità, occorre definire che cosa siano le organizzazioni criminali, come sono organizzate e di quali attività si occupano.
La criminalità organizzata può essere definita come l’insieme delle attività criminose compiute da organizzazioni criminali, perciò occorre definire prima di tutto queste ultime. Dunque, “è sicuramente definibile come organizzazione criminale quel gruppo strutturato, esistente per un periodo di tempo, composto da più di due persone che agiscono di concerto, al fine di ottenere, con l'esercizio della funzione intimidatoria, direttamente o indirettamente, un vantaggio finanziario o altro vantaggio materiale, e che pregiudica seriamente la coesione economica e sociale dell'Unione europea e dei suoi Stati membri, e di conseguenza lo stesso mercato unico” .
I principi guida si rifanno al concetto di “omertà”, che significa solidarietà, silenzio tra due o più persone appartenenti ad una stessa categoria o, in questo caso, organizzazione. Ogni membro è tenuto al segreto e così anche tutti coloro che subiscono il volere dell’organizzazione, per il timore di peggiorare la situazione. Le organizzazioni criminali si possono suddividere in tre categorie: le bande di strada, anche se piccole, avendo una struttura gerarchica sono considerate come organizzazioni criminali, le organizzazioni nazionali, che estorcono denaro a commercianti e ad imprenditori e corrompono i politici, e le organizzazioni internazionali, che gestiscono enormi traffici di armi, droga ed esseri umani. 
Quando si parla di criminalità organizzata, spesso si tende ad associare questo termine al concetto di mafia, ovvero alla forma di crimine organizzato presente sul suolo italiano, ma non ci si rende conto che, sparse in tutto il globo, da continente a continente, le organizzazioni criminali internazionali sono molteplici, come avremo modo di vedere all’interno di questo progetto quando parleremo delle organizzazioni criminali più conosciute a livello mondiale. (capitolo 4). Oggigiorno sono attive nei paesi sviluppati diciotto grandi strutture criminali internazionali.
Per quanto concerne l’aspetto della struttura interna di queste organizzazioni, la loro organizzazione può variare. Ad esempio, in Italia abbiamo le organizzazioni criminali come la Camorra e la ‘Nrangheta, che sono formate da clan, molto numerosi che si suddividono il territorio nazionale mentre, in Asia, la Yakuza giapponese è suddivisa in famiglie e La Triade cinese ha una struttura interna gerarchica a piramide.

\newpage

\input{Capitoli/Capitolo3.tex}
La criminalità organizzata esiste da molti anni. Secondo James Finckenauer, professore associato della “School of Criminal Justice”, la prima forma di crimine organizzato risale al 60 a.C, con Clòdio Pulcro. Clòdio Pulcro fu un politico romano, figlio di Appio Claudio Pulcro, nemico di Cicerone che, dopo essere riuscito a farlo esiliare e dopo avergli bruciato la casa, ricevette in avversario Milone. Entrambi erano a capo di bande armate, che possono essere considerate come delle vere e proprie organizzazioni criminali. Alla fine del conflitto fra i due, Milone rimase ucciso in un conflitto fra le due bande armate. Per quanto riguarda la storia della criminalità organizzata moderna, dobbiamo spostarci più avanti con gli anni, nel 1600, quando pirati, briganti e banditi attaccavano le vie del commercio, saccheggiando i carichi di beni e materie prime e influendo notevolmente sull’aumento del costo di questi ultimi, siccome i beni che venivano saccheggiati venivano poi riveduti ad un prezzo più alto, a vantaggio dei criminali. Il centro della pirateria nel XVII secolo era sicuramente la vecchia Port Royal, in Giamaica. Il criminologo Paul Lunde afferma che “la pirateria e il banditismo sono stati per il mondo pre-industriale quello che la criminalità organizzata è oggi per la società moderna" . Egli sostiene appunto che si tratta sempre di organizzazioni criminali vere e proprie, ma appartenenti a due epoche completamente differenti. Inoltre, sempre all’interno del suo libro, egli afferma che "i conquistatori barbari, sia Vandali, che Goti, Normanni, o le orde turche e mongole, non sono normalmente considerate gruppi criminali organizzati, ma hanno in comune con le organizzazioni criminali di successo molte caratteristiche” . Questi gruppi di persone agivano al rispetto delle proprie leggi, con l’uso della violenza e dell’intimidazione, proprio come ai giorni nostri. Le organizzazioni criminali moderne sono una semplice evoluzione di questi primi gruppi criminali, dal momento in cui agiscono e si comportano seguendo le stesse regole, per conseguire un obiettivo di natura economica.

\newpage

\section{Il legame con la tratta di esseri umani e con la prostituzione}

Le organizzazioni criminali internazionali si differenziano le une dalle altre non solo grazie al loro paese d’origine, ma soprattutto per le attività principali che praticano e che le caratterizzano. Tra le attività più comuni e frequenti svolte vi sono i traffici illegali di esseri umani, droga e armi, la tratta di esseri umani, la prostituzione, il gioco d’azzardo, gli omicidi e gli appalti (caratteristici delle mafie italiane, quali la Camorra, la ‘Ndrangheta e Cosa Nostra). All’interno di questo progetto si andrà ad analizzare solamente due di queste attività, ovvero la tratta di esseri umani e la prostituzione.

\subsection{La tratta di esseri umani}
La tratta di esseri umani, o tratta di persone, è un termine che indica “il reclutamento, il trasporto, il trasferimento, l’alloggio o l’accoglienza di persone, con la minaccia di ricorso o con il ricorso alla forza o con altre forma di costrizione, attraverso il rapimento, la frode, l’inganno, l’abuso di autorità o di una situazione di vulnerabilità, o attraverso l’offerta o l’accettazione di pagamenti o di vantaggi per ottenere il consenso di una persona che ha l’autorità su di un’altra a fini di sfruttamento” . Come è possibile dedurre dalla definizione, questa attività è illegale, e proprio per questo motivo viene gestita da organizzazioni criminali. Il suo giro d’affari è significativo; si parla di 150 miliardi di dollari, dei quali 100 miliardi provengono dalla tratta degli africani. Alla mafia nigeriana una donna trafficata frutta la cifra di 60 mila euro, e tenendo in considerazione che in Italia la mafia nigeriana è coinvolta nella tratta di 100 mila donne, la cifra ricavata ammonta a 600 milioni di euro. Le varie organizzazioni criminali che operano in questo settore sono strutturate, molto agguerrite e con un potere tale da esercitare un controllo totale nel loro paese, nel paese dal quale i soggetti della tratta partono e nei paesi di transito e destinazione. Le organizzazioni criminali protagoniste sono quella nigeriana, quella albanese, quella rumena, quella maghrebina, quella cinese e quella bulgara. La meta principale è l’Italia, perché si trova al centro dell’Europa. Dunque i soggetti di questa tratta arrivano in Italia e, in seguito, si spostano per poter raggiungere i paesi del nord Europa. La centralità dell’Italia non è l’unica motivazione, occorre aggiungere la facile raggiungibilità e vicinanza.

\newpage

\subsection{Lo sfruttamento della prostituzione}
Lo sfruttamento della prostituzione “consiste nel fatto di chi approfitti dei guadagni, in denaro o in un’altra utilità, purché economica, realizzati attraverso l’attività di prostituzione altrui” . Nel tempo questa attività criminosa è diventata un vero e proprio business. Si pensi solamente al fatto che un “magnaccia”, ovvero colui che sfrutta l’attività delle prostitute, guadagni, per ogni ragazza, 110 mila euro all’anno. Se si pensa che ogni protettore gestisce una decina di ragazze, il ricavo giornaliero ammonta a 9000 euro. Il profitto conseguito è considerevole se si considera il fatto che i costi sono molto contenuti. Il prezzo che si paga per l’acquisto una ragazza è variabile; oscilla tra i 300 e i 500 euro. Il tempo necessario per coprire questi costi risulta essere davvero limitato (le spese si possono tranquillamente coprire nel giro di un giorno). A differenza delle altre attività criminali come possono essere i vari traffici, di droga o armi, quella della prostituzione risulta essere la più redditizia, perché questo servizio piò essere venduto più volte. 
La donna viene considerata come oggetto, e non come un essere umano. Non stupisce che viene utilizzato spesso il termine “merce” per indicarle. È proprio per questo motivo che si parla di un vero e proprio mercato, dove queste donne vengono acquistate e sfruttate sessualmente dalle varie organizzazioni criminali, grandi o piccole che siano. Tra le varie organizzazioni che gestiscono questo business troviamo La Triade cinese, la Solncevskaja Bratva russa e la Yakuza giapponese, organizzazioni estremamente solide, con un controllo che si estende a livelli continentali.

\newpage

\section{Quali organizzazioni criminali internazionali gestiscono la tratta di esseri umani e la prostituzione}
\input{Capitoli/Capitolo5.1.tex}
\input{Capitoli/Capitolo5.1.1.tex}
La mafia giapponese, conosciuta più comunemente con il nome di Yakuza, domina da secoli il “paese del sol levante” ed è in costante crescita, crescita agevolata dalla notorietà e dall’immunità che questo complesso di organizzazioni criminali ha acquisito con il tempo in Giappone. Per risalire alle origini di tutto, dobbiamo per l’appunto spostarci di qualche secolo indietro. Tutto iniziò nel XV secolo, ai tempi dei feudi. Attorno al 1612 nascono le prime bande organizzate che godevano di un grande consenso popolare (chiamate “machi-yakko”, ovvero servitori del popolo), con l’obiettivo ultimo di contrastare i samurai, che a quei tempi seminavano il terrore. Tra queste bande, le più conosciute erano i Tekiya e i Bakuto. I Tekiya erano dei venditori ambulanti che difendevano i loro interessi dalla famiglia Tokugawa (1542-1612), ovvero i signori incontrastati del Giappone. Con il passare del tempo però, i Tekiya da venditori ambulanti si trasformarono in truffatori ambulanti, che ingannano e truffano gli abitanti dei villaggi. Il ruolo dei Bakuto fu totalmente diverso da quello dei Tekiya illustrato sopra. Infatti, i Bakuto basavano la loro attività principale sul controllo del gioco d’azzardo. Il nome che oggi attribuiamo alla mafia giapponese si deve ad un gioco di carte diffuso nel paese ai tempi dei Bakuto: l’hanafuga (anche chiamato “il gioco dei fiori”). Infatti, la parola Yakuza è formata da tre sillabe, ovvero dalla combinazione delle tre carte perdenti del gioco (8-9-3, “ya-ku-sa”). Ai Bakuto si devono anche le tradizioni mafiose come il dito tagliato (utilizzato come gesto riparatore) e i numerosissimi tatuaggi caratteristici diffusi su tutto il corpo (utilizzati per indicare l’appartenenza alla famiglia mafiosa “ikka” o “gumi”). Grazie ai legami acquisiti con l’economia, durante la seconda metà dell’Ottocento, la Yakuza gode di una grande protezione. Essa si allea con gli ultranazionalisti giapponesi e partecipa alla sanguinosa campagna che prese il nome di “governo per omicidio”, caratterizzata da numerosi omicidi di personaggi appartenenti alla sfera politica. Infine, grazie alla loro occupazione durante il periodo della seconda guerra mondiale, la Yakuza venne tollerata anche dall’america acquisendo molta forza e crescendo sempre di più in maniera esponenziale (si dice che «le sciabole si trasformarono in pistole»). 

\input{Capitoli/Capitolo5.1.2.tex}
La Yakuza conta all’incirca da 80'000 membri e la sua struttura organizzativa è costituita da famiglie (lo stesso modello delle mafie italiane come Camorra e ‘Ndrangheta), da un complesso di organizzazioni criminali, anche definite con il nome di “sindacati”. Queste famiglie (in giapponese “ikka” o “gumi”) sono indipendenti le une dalle altre e hanno una struttura gerarchica verticale, dove a capo vi è il padre (“oyabun”), a cui i figli (“wakashu”) devono sempre mostrare rispetto e fedeltà. I membri all’interno della classica famiglia mafiosa giapponese però non sono solamente questi due elencati sopra.
La figura a sinistra rappresenta quella che è la struttura classica di un sindacato. Come già anticipato sopra, i membri della famiglia non si limitano ad essere solamente il padre e il figlio, ma ve ne sono altri, anch’essi di uguale importanza. Tra questi troviamo ad esempio il “saiko-komon”, ovvero un consigliere supremo, colui che stringe i rapporti con gli interlocutori esterni, che è in contatto diretto con gli avvocati e che si occupa dell’amministrazione, il “waka-gashira”, ovvero il figlio prediletto e il “shatei-gashira”, ovvero il leader dei fratelli del padre “oyabun”, scelto da quest’ultimo basandosi sull’anzianità, sull’ideologia e sul merito.Le famiglie appartenenti a questo grande complesso mafioso che prende il nome di Yakuza sono molte, ma la più conosciuta, nonché la più grande, la più potente e la più pericolosa è la “Yamaguchi Gumi”, famiglia che ha sede a Kobe.

\input{Capitoli/Capitolo5.1.3.tex}
La Yakuza è un’organizzazione molto grande, che si estende in tutto il territorio asiatico e anche a livello mondiale. Proprio per questo motivo, un’estensione così vasta sul territorio richiede un business che frutti una grande quantità di denaro a quest’organizzazione criminale. Per l’appunto, la Yakuza basa il suo business principale sullo sfruttamento della prostituzione. Questo fenomeno si sviluppa in Giappone nel periodo a cavallo tra gli anni ’70 e gli anni ’80, quando le varie società giapponesi organizzavano i viaggi aziendali per i propri dipendenti nel nord e nel sud-est asiatico. La Yakuza è riuscita a monetizzare, a ricavare denaro da questi viaggi, offrendo un vero e proprio tour sessuale per i numerosi lavoratori in viaggio. Come scrivono David E. Kaplan e Alec Dubro all’interno del loro libro sulla mafia giapponese intitolato “Yakuza: Japan’s Criminal Underworld”, dopo aver “esportato gli uomini”, è arrivato il momento di “Importare le donne”. È infatti attorno agli anni 2000 che in Giappone si contavano all’incirca 100 mila prostitute straniere. Come attestano i due autori all’interno del loro libro, “Alla fine degli anni ’90 una sessione con una prostituta straniera costa intorno ai 200 dollari. Con 100mila prostitute straniere impiegate in almeno una prestazione al giorno ogni anno entrano 7,3 miliardi di dollari nelle tasche di protettori, proprietari di bar, intermediari e, infine, delle stesse prostitute”. Inolte, la Yakuza per ingaggiare nuove prostitute in maniera tale da alimentare questo giro criminale, si affida ad una rete di reclutatori che si muove nei paesi più poveri come Thailandia e Filippine, per offrire delle somme di denaro alle famiglie più povere in cambio dell’acquisto di una loro figlia. Da questi dati si capisce dunque come lo sfruttamento della prostituzione sia un business estremamente redditizio per la mafia giapponese e, allo stesso tempo, di come esso sia estremamente crudele.

\newpage

\input{Capitoli/Capitolo5.2.tex}
\input{Capitoli/Capitolo5.2.1.tex}
La Triade cinese, l’enorme e potente mafia cinese, ha origine attorno alla seconda metà del XVII secolo, dove abbiamo la contrapposizione tra i sostenitori della dinastia dei Ming contro i loro invasori, i suddetti Manciù (anche denominati “la dinastia Quing” o “Ch’ing), che furono una popolazione insediatasi nella regione nord-orientale dell’Asia, la Manciuria. Questi ultimi presero il potere nel 1644. I perdenti del confronto, ovvero la dinastia dei Ming, raccolse i propri combattenti, i propri partigiani e si riunirono tutti quanti all’interno di una società segreta molto numerosa che fu fondata da dei monaci guerrieri. Questa società segreta veniva chiamata Hong Mon, dove Hong stava ad indicare il nome di uno dei pretendenti al trono, pretendenti che venivano sostenuti dai vari guerrieri Ming. Il motto della Hong Mon era “Fan Qing fu Ming!”, che tradotto in italiano significava “Deponete i Qing e restaurate i Ming!”. I primi associati a questa organizzazione segreta, che vedremo diventare una vera e propria organizzazione criminale, stabilirono il loro luogo di incontro all’interno del monastero buddhista di Shaolin, che si trova nella provincia di Henan. All’interno di questo monastero veniva insegnata una tecnica di combattimento che si effettuava a mani nude, conosciuta ancora oggi: il Kung-fu. Nel 1674 il monastero di Shaolin, sede della Hong Mon, venne distrutto dall’esercito rivale dei Qing. Solamente cinque guerrieri riuscirono a sopravvivere a questo attacco. Questi guerrieri erano conosciuti come “le tigri Shaolin”, che a seguito del disastro si trasferirono nel Guangdong, una provincia sud-orientale della Cina. In questa provincia vengono fondate numerose nuove società, nuove organizzazioni, che si unificarono all’interno di essa. Gli attacchi da parte della dinastia rivale dei Qing però continuano e nel 1717 bandirono anche il cristianesimo. Questo però non impedì ai seguaci delle “tigri dello Shaolin” di costituire delle nuove organizzazioni, seguendo il modello della società che ha dato inizio a tutto, ovvero sull’impronta della Hong Mong. Seguentemente, dopo l’unione tra queste nuove organizzazioni e la “Setta del loto bianco” (organizzazione segreta e a tratti religiosa, definita anch’essa un’illustre antenata della grande mafia cinese) nasce la Triade, che viene chiamata cos’ per il triangolo simboleggiante la relazione armoniosa fra la terra, il cielo e l’uomo, posto nell’ideogramma Hong.

\input{Capitoli/Capitolo5.2.2.tex}
La struttura organizzativa della Triade è costituita da un modello di tipo gerarchico verticale, che ripercorre le caratteristiche del modello precedente riferito alla Yakuza. L’unica cosa che cambia è che questo tipo di modello identifica ogni ruolo all’interno dell’organizzazione criminale tramite un numero. Come possiamo facilmente notare da questa immagine e come già citato prima, ogni ruolo all’interno della struttura organizzativa è riconosciuto tramite un numero, numero seguito dal nome del ruolo che si sta ricoprendo. Si parte sempre dall’alto, laddove troviamo i ruoli di maggior importanza, fino ad arrivare ai ruoli più umili e meno importanti, posti alla fine della struttura. Al vertice di quest’ultima troviamo dunque il numero 489, detto “Testa del Drago” o “Signore della Montagna” (il leader, colui che ha il controllo di tutto: il boss). Seguentemente, subito sotto al capo supremo troviamo i numeri 438, detti anche “L’Avanguardia”, colui che lavora assieme al “Maestro d’Incenso” (o “Capo d’Incenso”, che si occupa del cerimoniale, ovvero di tutto l’insieme di regole da applicare durante le cerimonie) e il “Vice-Signore della Montagna”. Scalando di un gradino troviamo invece il numero 415, detto anche “Il Ventaglio di Carta Bianco” (colui che si occupa del lato economico, effettuando la ricerca delle varie risorse e proponendo delle soluzioni, viste come dei consigli), il numero 426, detto anche “Il Guerriero del Polo Rosso” (colui che si occupa del settore militare e della giustizia interna) e il numero 432, detto anche “Il Sandalo di Paglia” (colui che si occupa della trasmissione delle informazioni). Infine, ai piedi della struttura, troviamo come già anticipato in precedenza, i ruoli più umili e meno importanti: il numero 49, detto anche “Il Membro Ordinario” (membri generici ai quali non è stata assegnata nessun incarico particolare, come un semplice soldato) e l’ultimo ruolo, le cosiddette “Lanterne Blu” (comprende tutti coloro che non sono ancora stati integrati alla gang, ma sono considerati come dei semplici “associati”). 
Fino alla metà degli anni ’90, la gang più grande presente sul territorio cinese che trattava con la Triade era la “14 K” (dove la lettera “K” sta a significare i carati), formatasi a seguito della fine della seconda Guerra Mondiale e della Guerra Civile cinese.

\input{Capitoli/Capitolo5.2.3.tex}
Anche per quanto riguarda la Triade cinese, il business più redditizio è quello dello sfruttamento della prostituzione. A differenza della Yakuza però, la Triade, tramite questa attività criminale è riuscita ad estendersi notevolmente a livello globale. Principalmente, il business dello sfruttamento della prostituzione controllato dall’organizzazione criminale cinese, oltre che in Cina, risulta essere molto redditizio in Italia, dove quest’ultima si è espanda ed esercita un controllo molto forte. Sono molti infatti i bordelli che la Triade ha organizzato in Italia, all’interno delle maggiori città. All’interno di questi bordelli, le ragazze (giovani soprattutto, attorno ai 19 anni) vendono il proprio corpo in cambio di una somma di denaro molto bassa. Si stima che, giornalmente, ogni prostituta garantisca alla mafia cinese all’incirca un migliaio di euro, tenendo in considerazione anche il fatto che l’80\% dei ricavi rimane nelle mani della ragazza. Per adescare clienti, la Triade gestisce il proprio commercio tramite dei veri e propri call center, situati all’interno delle attività commerciali cinesi, che si occupano di indirizzare il cliente e di deviare le telefonate, in base al servizio richiesto. Le forze dell’ordine italiane hanno effettuato delle indagini per cercare di fermare questo fenomeno e da queste ultime è emerso che le prostitute presenti all’interno di questi bordelli gestiti dalla Triade provengono per la maggiore da Hong Kong. Esse arrivano in Italia in aereo con il visto turistico e, al termine dei due mesi consentiti, diventano delle clandestine. Per vanificare gli interventi delle forze dell’ordine, l’organizzazione criminale utilizza gli spazi di cui è proprietaria a rotazione, in maniera tale da spostare l’attività criminale da un posto all’altro costantemente, non permettendo a nessuno di avere dei punti di riferimento su dove basare le indagini, che sono in costante crescita, soprattutto nei territori di Milano, Firenze, Roma e Palermo.

\newpage

\subsection{Organizacija (Fratellanza Solncevskaja), Russia}
\input{Capitoli/Capitolo5.3.1.tex}
Prima di passare alla storia della Fratellanza Solncevskaja, occorre fare un passo indietro e parlare della creazione dell’”Organizacija”, ovvero della potentissima mafia russa in generale, poiché è da quest’ultima che si svilupperà poi l’organizzazione criminale di cui tratta questo capitolo.
La mafia russa (“Organizacija”) ha una storia molto lunga, che si può delineare tramite la suddivisione di essa in due periodi: prima e dopo la caduta dell’Unione Sovietica. Le prime organizzazioni criminali (erano piccole e non avevano una struttura interna ben delineata) nacquero infatti nell’URSS durante gli anni trenta, a causa delle scarse condizioni economiche. Queste organizzazioni, agevolate dal clima corrotto dell’epoca, presero il controllo del mercato nero e crearono un sistema economico nascosto ma parallelo allo stato. La figura vera e propria del mafioso russo come la conosciamo oggi e il concetto di Organizacija nacquero però nei gulag (luoghi dove Stalin rinchiudeva i criminali con l’obiettivo ultimo di arginare i loro traffici), dei veri e propri campi di addestramento, dove si eseguivano i riti di iniziazione per entrare a far parte di queste organizzazioni criminali, e ciò avrebbe arricchito il loro status di uomini potenti e pericolosi. I gulag permisero inoltre la formazione dei cosiddetti “Vory v Zakone”, che tradotto significa “i ladri della legge”. Alla caduta dell’unione sovietica si passò da un’economia di stampo comunista ad un’economia di mercato, che generò una privatizzazione delle industrie e delle proprietà dello stato, tutto questo seguito da una pesante inflazione dopo la liberazione economica. Molti uomini d’affari e, tra questi, anche i membri dell’Organizacija, approfittarono di questa situazione e si insediarono nei punti strategici del nuovo Stato grazie alla loro capacità di saper manipolare e ricattare. A metà degli anni ’80 venne creata la “Solncevskaja Bratva” (o “Fratellanza Solncevskaja”), anche comosciuta come “Brigata del Sole”, una tra le più pericolose organizzazioni criminali a livello mondiale e una delle gang più potenti della Russia. Essa venne fondata dal mafioso russo Syergyej Mihajlov ‘Mikhas e dal suo compagno criminale di Viktor Averin. Essi volevano basarsi più su uno stile occidentale, scostandosi dunque da quello che era il modello dei “ladri della legge”, ignorando le linee guida lasciate da questi criminali russi.

\input{Capitoli/Capitolo5.3.2.tex}
La struttura organizzativa della mafia russa (Organizacija) è facilmente divisibile in tre livelli. Il primo livello vede come protagoniste 10/15 piccoli gruppi, piccole gang che risultano essere indipendenti dall’azienda, ma rimangono comunque affiliate a quest’ultima. All’interno del secondo livello possiamo trovare dei gruppi più grandi, formati da oltre 200/300 membri. Di questi gruppi se ne contano all’incirca 500 sparsi su tutto il territorio russo, occupandosi del controllo delle gang più piccole. Per finire invece, all’interno del terzo livello troviamo i cosiddetti (citati sopra) “ladri della legge” (in russo “Vor v Zakone”), che sono all’incirca 150. Essi sono la gang con più potere economico dell’intera mafia russa, siccome quest’ultimo livello è composto per lo più da avvocati, ingegneri, medici e politici. Grazie a questo grande potere economico, sono loro che si occupano delle più onerose operazioni finanziarie.

\input{Capitoli/Capitolo5.3.3.tex}
Oltre ai vari omicidi commessi nel periodo degli anni ’80 e allo spaccio di droga, che copre una grande fetta del business illegale, la mafia russa (Organizacija) e in particolare la sua maggiore organizzazione criminale, la Fratellanza Solncevskaja, è anch’essa come le altre due analizzate all’interno dei capitoli precedenti, spesso coinvolta in operazioni illegali di tratte di esseri umani e di sfruttamento della prostituzione.

\section{La criminalità organizzata in Svizzera}
L’ufficio federale di polizia (UFP) ha pubblicato un rapporto sulla situazione della criminalità organizzata in Svizzera, che ci permette di avere una visione ben chiara della vicinanza di questo fenomeno e di capire quanto questa si sia espansa. Dal rapporto pubblicato le prime informazioni che emergono sono che le organizzazioni criminali principali presenti sul suolo Svizzero hanno origine intrattengono dei rapporti con la criminalità organizzata russa. Dalle varie ricerche effettuate e dai vari rapporti risulta che dei 687 cittadini svizzeri e stranieri che hanno dei rapporti diretti o indiretti con la criminalità organizzata russa, 153 di questi sono sospettati di aver commesso degli atti criminali. Lo stesso discorso vale anche per le aziende. Di quelle 194 ditte che offrono un servizio commerciale o finanziario che intrattengono dei rapporti diretti o indiretti con la criminalità organizzata russa, 90 di queste sono coinvolte in atti criminali. Gli atti criminali più frequenti che vengono praticate sul suolo Svizzero sono il riciclaggio di denaro e la frode. Infatti, il settore maggiormente colpito dalle attività criminali è quello finanziario. Le organizzazioni criminali utilizzano la Svizzera come base per le loro transazioni di denaro e per il riciclaggio di denaro sporco. Le regioni della Svizzera che sono particolarmente assoggettate a questi tipi di crimini sono i cantoni di Zurigo, Ginevra e Ticino, ovvero le maggiori piazze finanziarie svizzere. Per quanto riguarda l’attività di riciclo di denaro sporco, il Ticino è sempre stata una zona molto gettonata dalle organizzazioni criminali italiane come la Camorra o la ‘Ndrangheta, e questo non è dato solamente dal criterio geografico (dalla vicinanza del Ticino con il confine italiano), ma è dato anche dal fatto che in Svizzera la situazione economica sia migliore di quella italiana. Il riciclaggio di denaro sporco in Svizzera permetteva dunque di far perdere le tracce di quest’ultimo e anche della sua origine illecita.

\section{La tratta di esseri umani e la prostituzione in Svizzera}
In Svizzera, come in tutto il mondo, l’attività più classica e comune praticata dalle varie organizzazioni resta sempre quella della tratta delle donne dal loro paese d’origine fino al paese dove eserciteranno illegalmente la prostituzione. Il numero dei club, dei bar o delle case dove il 95\% delle donne (che si presentano come delle turiste) si prostituisce illegalmente in Svizzera è in aumento. Queste donne provengono per la maggiore dai paesi dell’est ed è molto difficile che esse rilascino delle informazioni alla polizia in merito al modo in cui sono state reclutate, o in merito a qualche organizzazione criminale. Risulta dunque difficile comprendere in che maniera queste donne sono state reclutate e quali organizzazioni criminali gestiscono questo giro di attività illegali.
Per quanto riguarda le condanne legate al turismo sessuale, queste ultime, se poste a confronto con ad esempio le condanne per reati sessuali commessi sui bambini in relazione e non con la pornografia, risultano essere nettamente inferiori. Questo perché le procedure penali legate ai casi, che comprendono l’assunzione delle prove e le inchieste e le indagini all’estero (nei paesi dai quali queste donne provengono, dai quali sono state trafficate) risultano essere molto costose. Per questo motivo l’ufficio federale di polizia (UFP) ha deciso di intensificare la collaborazione con alcune organizzazioni non governative (ONG), al fine di riuscire ad aumentare il livello di informazioni, in maniera tale da rendere meno oneroso il costo delle indagini.

\newpage

\section{Differenze fra le varie organizzazioni criminali}

\newpage

\input{Introduzione_conclusione_bibliografia/Conclusione.tex}

\newpage

\input{Introduzione_conclusione_bibliografia/Bibliografia.tex}

\end{document}
